\documentclass[conference]{IEEEtran}
\IEEEoverridecommandlockouts
% The preceding line is only needed to identify funding in the first footnote. If that is unneeded, please comment it out.
\usepackage{cite}
\usepackage{amsmath,amssymb,amsfonts}
\usepackage{algorithmic}
\usepackage{graphicx}
\usepackage{textcomp}
\usepackage{multirow}
\usepackage{xcolor}
\usepackage{listings}
\def\BibTeX{{\rm B\kern-.05em{\sc i\kern-.025em b}\kern-.08em
    T\kern-.1667em\lower.7ex\hbox{E}\kern-.125emX}}


\begin{document}

\title{Floating Point Implementation\\
    UKY EE 480}

\author{\IEEEauthorblockN{Grant Cox}
\textit{University of Kentucky}\\
grant.cox@uky.edu
\and
\IEEEauthorblockN{Josh Carroll}
\textit{University of Kentucky}\\
Josh email
\and
\IEEEauthorblockN{3rd}
\textit{University of Kentucky}\\
email}

\maketitle




\begin{abstract}
    The final component of the PinKY archetecture was to be a "mutant" 16-bit floating point module (FPU). The FPU performs conversions, multiplications, reciprocals, additions, and subtractions on floats with 1 sign bit, 8 exponent bits, and 7 mantissa bits. This is a multicycle design using a state machine in Verilog. It was simulated and tested in Icarus Verilog with GtkWave.
\end{abstract}

\section{Approach}
    \subsection{FTOI}
    \subsection{ITOF}
    \subsection{MULF}
    \subsection{RECF}
    \subsection{ADDF}
    \subsection{SUBF}

\section{Issues}
    It don't work yet


\end{document}